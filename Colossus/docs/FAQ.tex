\documentclass{article}
\usepackage[latin1]{inputenc}
\usepackage{html}

\begin{document}

% $Header$

\title{Frequently Asked Questions for Colossus}

\author{David Ripton}

\maketitle

\begin{itemize}

\item[Q] Why does the game sometimes prompt me twice for recruits?

\item[A] Look closer. It's asking for the recruit and then for the recruiter(s).
 Do you want to recruit that cyclops with your other cyclops, or with
 your two gargoyles? If like most people you don't care, then turn on 
 ``auto pick recruiter'' and the computer will just pick one for you.

\item[Q] Why is the angel X'd out in the summon dialog?  I have one.

\item[A] Because you haven't selected the donor stack with the angel.

\item[Q] How to I make a legion spin around in a circle back to the original
 hex when I roll a 6? 

\item[A] Just click on the chit twice.

\item[Q] What does the Antialias option do?

\item[A] It makes the graphics a bit smoother (look closely at the hexside
 edges while you turn it on and off), but this takes some CPU cycles. I
 recommend turning it on if you have a fast computer and leaving it off
 if you have a slow computer. 

\item[Q] When is network play going to be done?

\item[A] It's done.  Now we just have to make it more robust.


\item[Q] Why is the AI so dumb?

\item[A] Titan has a huge branching complexity.  So brute force searching
through all possible moves several turns ahead isn't possible.


\item[Q] What's the difference between SimpleAI and CowardSimpleAI and
 A Random AI?

\item[A] SimpleAI does a straightforward one-ply lookahead for MasterBoard
 moves. It works fairly well.  CowardSimpleAI is the same code with some
 constants changed to make it less aggressive.  A Random AI just chooses one
 of the available AIs at random -- this will be a more interesting option
 when there are more viable alternatives.


\item[Q] What's the ``Load External Variant'' button do? 

\item[A] The maps and recruit trees and stuff used to be hardcoded. Romain
 pulled them out into data files, which means that you can customize parts 
 of the game by making new data files instead of changing code. If that 
 interests you, see
 \htmladdnormallink{Files Formats}{../FileFormat/index.html}
 and the
 \htmladdnormallink{Variant-HOWTO}{../Variant-HOWTO/index.html}.
 A bunch of variants are now included with the game.

 ``Load External Variant'' allows for loading an external variant
 (not supplied with the game), whereas the pop-up menu allows
 for loading one of the standard variants (supplied with the game).

\item[Q] Why, in the Caretaker's Stack, some creatures have \texttt{--}
 instead of a number ?
\item[Q] Why, in the Caretaker's Stack, some creatures aren't red-crossed
 when there's 0 left ?

\item[A] Same reason: those creatures are ``immortal'', i.e. when they're
 killed on the batllefields, they're put back in the Caretaker's Stack
 and not in the Graveyard (for instance, in traditional Titan, they're
 Angel, Archangel, Warlock, Guardian and Titan).

\item[Q] What's the option ``Slowing is cumulative'' ?

\item[A] In regular Titan, there's never two slowing reasons together
(you don't find a bramble up a slope, for instance). In Variants this is
possible, so we must decide what to do: do you need 2 (only one slow) or 3 (count them both) movement points to go to an up-slope bramble (assuming you're
non-native) ? Without this option, 2. With this option, 3. We don't decide,
you do :-)

\item[Q] What's the option ``Always allows one hex'' ?

\item[A] This is necessary after introducing the aforementioned option
``Slowing is cumulative''. What if a 2-skill creature want to enter an
up-slope bramble ? Without this option, it's impossible. With this option,
any creature can enter any non-impassible hex as its first move (it will only
move one hex), so our 2-skill creature can enter our up-slope bramble
if it hasn't moved yet.

\item[Q] How can I help?

\item[A] Bug reports are great. Detailed bug reports delivered via the 
 SourceForge bug tracker are even better. 
 
 If you want to contribute code, make sure that you're starting 
 from the latest source (so pull from CVS). Please read and follow
 the \htmladdnormallink{CodingStandards}{../CodingStandards/index.html}
 so your code is easier to merge. Join and send mail to the dev mailing
 list at SF so we know what you're up to. Beyond that, just code whatever
 you want and send patches when it works.

\end{itemize}

\end{document}
