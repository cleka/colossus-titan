\documentclass{article}
\usepackage[latin1]{inputenc}
\usepackage{html}

\begin{document}

% $Header$

\title{The Colossus Battlelands Randomizer}

\author{Romain Dolbeau}

\maketitle

\section{BattlelandsRandomizer}

BattlelandsRandomizer is a (very crude) tool to randomly
create Battlelands for use with the
``\htmladdnormallink{Colossus}{http://colossus.sf.net/}'' Game.
It has the same requirements as Colossus itself, see
\htmladdnormallink{Colossus README}{../../README/index.html}
and
\htmladdnormallink{Colossus build-HOWTO}{../../build/index.html}.

To create the program, just type ``make datatools'' or ``and tools''.
See the
\htmladdnormallink{Colossus build-HOWTO}{../../build/index.html}
for more details.
This should create all the datatools for Colossus, including
BattlelandsRandomizer.

BattlelandsRandomizer use an input file that describe
the properties of the future Battleland, and generate
one random Battleland. At the moment, it's a purely
static, text-based process.

The Randomizer used can also be accessed from inside
\htmladdnormallink{BattlelandsBuilder}{../BattlelandsBuilder/index.html},
in a much more user-friendly way. An input file (described below) is
still needed.

Examples of random input file can be found in the
``Random'' subdirectory.

Properties are line of one of the three forms:

\begin{enumerate}

\item \texttt{AREA $\langle$list\_of\_hex$\rangle$ HAZARDS $\langle$list\_of\_hazards$\rangle$}
\item \texttt{LABEL $\langle$label$\rangle$ = $\langle$list\_of\_hex$\rangle$}
\item \texttt{PAIR $\langle$pair\_of\_hazard$\rangle$ HEXSIDES $\langle$list\_of\_hexsides$\rangle$}

\end{enumerate}

where

\begin{itemize}

\item \texttt{AREA}, \texttt{HAZARDS}, \texttt{PAIR}, \texttt{HEXSIDES} and \texttt{LABEL} are the strings literals ;
 = is the character literal '='.

\item $\langle$list\_of\_hex$\rangle$ is a list of hex from the following available forms:
\begin{itemize}
\item \texttt{"("$\langle$number$\rangle$,$\langle$number$\rangle$")"}
\item \texttt{"("$\langle$range$\rangle$,$\langle$range$\rangle$")"}
\item \texttt{"("$\langle$number$\rangle$,$\langle$range$\rangle$")"}
\item \texttt{"("$\langle$range$\rangle$,$\langle$number$\rangle$")"}
\item \texttt{$\langle$hex\_label$\rangle$}
\item \texttt{"("$\langle$label$\rangle$")"}
\item \texttt{SOMEOF"("$\langle$number$\rangle$,$\langle$list\_of\_hex$\rangle$")"}
\item \texttt{SURROUNDINGSOF"("$\langle$list\_of\_hex$\rangle$")"}
\item \texttt{SUBSTRACT"("$\langle$list\_of\_hex$\rangle$,$\langle$list\_of\_hex$\rangle$")"}
\end{itemize}
 where $\langle$number$\rangle$ is a number between 0 and 5, and $\langle$range$\rangle$ is of the form
 $\langle$number$\rangle$-$\langle$number$\rangle$
 i.e. a range of number between 0 and 5.
 $\langle$hex\_label$\rangle$ is simply the label displayed on the Battlemap (F4, C3,...)
 $\langle$label$\rangle$ is a label defined previously with a \texttt{LABEL} statement (see below)

 There are some predefined labels available, including :
\begin{itemize}
\item ``inside'' representing everything but the most outside Hex.
\item ``leftdefenseentry'', ``upperdefenseentry'' and ``lowerdefenseentry'',
 the 3 lines of 3 hexagons representing the defense entry lines
 (left, upper-right, and lower-right respectively).
\item ``anywhere'' representing the whole Battlelands.
\item ``leftover'' : special label representing all hexes that were in
 the most recent \texttt{AREA} statement, but didn't receive a Hazard.
\item ``usedup'' : special label representing all hexes that were in
 the most recent \texttt{AREA} statement, and did receive a Hazard.
\end{itemize}
 
 \texttt{SOMEOF} is a literal string ; result is a list of $\langle$number$\rangle$ hexes taken
 randomly from the $\langle$list\_of\_hex$\rangle$ in parameter.

 \texttt{SURROUNDINGSOF} is a literal string ; result is the list of hexes
 adjacent to any hex(es) in the $\langle$list\_of\_hex$\rangle$ in parameter. Note that
 all hexes present in the $\langle$list\_of\_hex$\rangle$ are excluded from the result.

 \texttt{SUBSTRACT} is a literal string ; result is the list of hexes that
 are in the first list but not in the second.

\item $\langle$list\_of\_hazards$\rangle$ is a list of Hazard from the following available forms:
\begin{itemize}
\item \texttt{$\langle$hazard\_letter$\rangle$(,$\langle$probability$\rangle$)?(,$\langle$elevation$\rangle$)?}
\item \texttt{$\langle$hazard\_letter$\rangle$(,$\langle$probability$\rangle$)?(,$\langle$probability\_elevation\_0$\rangle$,$\langle$probability\_elevation\_1$\rangle$,$\langle$probability\_elevation\_2$\rangle$)?}
\end{itemize}

 where $\langle$hazard\_letter$\rangle$ is a letter representing a
 Hazard ('r' for Brush, and so on, see BattleHex.java), $\langle$probability$\rangle$ is a float number
 representing the percentile of chance of the hazard being present
 (a numerical point \_must\_ be present for all probability), $\langle$elevation$\rangle$
 is a hard-coded elevation (integer between 0 and 2), and the three
 $\langle$probability\_elevation\_X$\rangle$ are the odds of the hazard being at elevation
 0, 1 and 2 respectively (it's better is the sum is 100., but they will
 be normalized to percentile values).
 Only the $\langle$hazard\_letter$\rangle$ is mandatory, the other parts are optionals.
 But if those other parts are present, they must respect the order
 above (first probability if present, the either the elevation or
 the elevation probabilities).

\item $\langle$label$\rangle$ is a lowercase-only string, that act afterwards as it was the list of
 Hex on the right side of the '=' character.

\item $\langle$list\_of\_hexsides$\rangle$ is alist of $\langle$hexside\_letter$\rangle$,$\langle$hexside\_probability$\rangle$

\item $\langle$pair\_of\_hazard$\rangle$ is a pair of "("$\langle$hazard\_letter$\rangle$,$\langle$elevation$\rangle$")". Elevation
 can be 0, 1, 2 or the character literal *, which means 'any elevation'.
 For all hexside between such a pair of hazard, all hexside from the
 $\langle$list\_of\_hexside$\rangle$ will be tried in order, until either one is
 present, or all have been tried (in which case no hexside will be
 used between the two hazards). Not that the default Hazard, the
 plains (character 'p'), can be used.

\end{itemize}

BattlelandsRandomizer will then, for each \texttt{AREA}, try to put the
Hazard in any one of the Hex in the \texttt{AREA}. An Hex will not receive
two hazards, but if an Hex is present multiple times in the same
\texttt{AREA}, it may receive one Hazard per occurence of the Hex. In this
case, only the last Hazard is used.

\texttt{AERA}s are processed in order, so an Hex that appears in multiple
\texttt{AREAS} may also receive multiple Hazards. In this case alsom
only the last Hazard is used.

Note that all hexsides use the same \texttt{PAIR}-defined probability. Only the
last \texttt{PAIR} statement is used for a given pair. For this purpose, all
elevation including * are different. The more specific statement will
be used (i.e. (s,0)(p,0) will be used instead of (s,*)(p,*) for
the hexside between a 0-height Sand and a O-height Plains).

\end{document}
