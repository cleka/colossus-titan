\documentclass{article}
\usepackage[latin1]{inputenc}
\usepackage{html}

\begin{document}

% $Header$

\title{Building Colossus}

\author{David Ripton, Romain Dolbeau}

\maketitle

Sorry this is so longwinded; I'm trying to be thorough in case 
someone who's not a Java programmer wants to make a small change.

\section{Tools needed}

All of these tools are free downloads, and we need to avoid making 
the game depend on tools that aren't.

\subsection{Java development kit}

You need a \texttt{JDK}, which includes a javac compiler, a java runtime
environment,  and various utilities like jar and javadoc.

The game should work with \texttt{JDK} 1.3 or newer. 1.3.1 or 1.4.0 seems to be
the most stable.  It's a good idea to keep a 1.3.x \texttt{JDK} around for 
compatibility testing.

Unless you have a fast, reliable, nailed-up net connection, also 
download the \texttt{JDK} docs, which are packaged separately.

\htmladdnormallink{Sun}{http://www.sun.com/} maintains the \texttt{JDK}
for MS Windows, Solaris Sparc and x86, and x86 Linux. See
\hypercite{Sun JDK}{}{}{SUNJDK}. I think that the
\htmladdnormallink{Sun}{http://www.sun.com/} \texttt{JDK} is currently
the best one on those three platforms.

The Blackdown \texttt{JDK} is the original port of
\htmladdnormallink{Sun}{http://www.sun.com/}'s Solaris \texttt{JDK}
to Linux. (\htmladdnormallink{Sun}{http://www.sun.com/}'s Linux
\texttt{JDK} in turn uses Blackdown code. So the two are very close cousins.)
If you have a non-x86 Linux box, you want Blackdown. See
\hypercite{Blackdown JDK}{}{}{BLACKDOWNJDK}.

\texttt{gcj} and \texttt{kaffe} don't have enough \texttt{GUI} support yet to
be useful for Colossus.

\begin{itemize}
\item[TODO:] More platforms. Please update this document if you use
platforms that I don't.
\end{itemize}

If your platform is not listed above, a starting point for finding a
\texttt{JDK} is \htmladdnormallink{http://java.sun.com/cgi-bin/java-ports.cgi}{http://java.sun.com/cgi-bin/java-ports.cgi}

\begin{itemize}
\item[TODO:] Do we need to show different environment variable syntax and
startup script lines on various platforms, shells, etc.? Ugh. Find a link.
\end{itemize}

Once you have installed your \texttt{JDK}, you probably want to set an
environment variable called \texttt{JAVA\_HOME} that points to its base
directory. Then you want to put \texttt{\$JAVA\_HOME}/bin in your
\texttt{PATH}. You also want to set \texttt{CLASSPATH} to include ``.''
(the current directory); you'll add stuff to it later as you add Java tools.

Not many tools directly use \texttt{JAVA\_HOME}, but it lets you easily
switch \texttt{JDK}s by changing one environment variable instead of all
the things that point to it.

\subsection{Other compilers}

\hypercite{Jikes}{Jikes}{}{JIKES} is Open Source very fast Java compiler
written in C++, originally
written by \texttt{IBM}. Its speed is very nice, and it has some other neat 
features like incremental compilation, more warnings, etc. However, 
jikes produces slightly different class files than javac, which 
sometimes causes problems. So I recommend using Jikes most of the time,
but making sure to do a full compile with javac as well before checking
in code.
Also, you may need to set the \texttt{JIKES\_PATH} environment variable so
Jikes can find your standard Java libraries. And you probably want Jikes in
your \texttt{PATH}.

Read the Ant docs for how to set the build.compiler property to 
switch between javac and Jikes.

\subsection{Build tools}

There are two main alternatives for building Colossus,
\hypercite{ant}{ant}{}{ANT} and make.

Ant is a newer cross-platform build tool used by many Java projects. 
It uses a file called build.xml. You will also need the corresponding
``optional.jar'' which contains extra features that we use, like the 
Java\texttt{CC} task. You need version 1.4.1 or newer. You'll need to set 
\texttt{ANT\_HOME} to its install directory, and you'll want
\texttt{\$ANT\_HOME}/bin in your \texttt{PATH}.

Make is the old-school Unix build tool. It uses a build file called
Makefile. (Colossus only has one, not nested Makefiles in each
directory.) Make is to Ant as C is to Java. It's fast, it's already
included on Unix boxes, it's portable only if you handle all the special
cases yourself, it's somewhat user-hostile, and a lot of people already 
know it because it's been around forever. Please try to use Ant instead, 
so that we can someday get rid of the Makefile and avoid double 
maintainence.

\subsection{Parser tools}

Colossus uses the \hypercite{JavaCC}{JavaCC}{}{JAVACC} parser generator
written by \htmladdnormallink{Sun}{http://www.sun.com/} and Matamata.
(Java\texttt{CC} has nothing to do with javac, despite the similar name.
It stands for Java Compiler Compiler.)

This turns .jj files into .java files, which are then compiled normally. 
You want version 2.1.

Direct access to the archive file:
\htmladdnormallink{http://www.webgain.com/trialware/javacc/JavaCC2\_1.zip}{http://www.webgain.com/trialware/javacc/JavaCC2\_1.zip}.

I set \texttt{JAVACC\_HOME} to its install location and then put
\texttt{\$JAVACC\_HOME}/bin in my \texttt{PATH}. (You can probably
see the pattern by now.)

You don't actually need to know Java\texttt{CC} unless you want to change
parser files or make new ones. You only need to install it and let build.xml
drive it for you.

\subsection{Source code control}

We use \hypercite{\texttt{CVS}}{\texttt{CVS}}{}{CVS}, with the repository 
hosted on \hypercite{SourceForge}{SourceForge}{}{SOURCEFORGE}.
\texttt{SF} requires \hypercite{ssh}{ssh}{}{SSH} or
\hypercite{OpenSSH}{OpenSSH}{}{OPENSSH} instead of the insecure pserve
as a transport mechanism, which complicates cvs setup a bit. There are a
lot of \texttt{CVS} + ssh setup directions on \texttt{SF}. If you can't
get it to work, ask for help.

If you're under Win32 (that's MS Windows 95+ or NT and later), see
\hypercite{PuTTY}{}{}{PUTTY} for a SSH/Telnet client. General SourceForge
documentation is at \htmladdnormallink{http://sourceforge.net/docman/?group\_id=1}{http://sourceforge.net/docman/?group\_id=1}.

\begin{itemize}
\item[TODO:] Insert \texttt{SF} \texttt{CVS} setup instructions, etc.
\end{itemize}

\subsection{Java Web Start}

If your \texttt{JDK} didn't include Java Web Start, you want that too, so 
that you can test that you haven't broken \texttt{JWS} compatibility. (This
is way too easy to do when you add resources.)
\htmladdnormallink{http://java.sun.com/products/javawebstart/}{http://java.sun.com/products/javawebstart/}. You want \texttt{javaws} in your
\texttt{PATH}.

(An alternate \texttt{JNLP} launcher, in case there is no \texttt{JWS} for
your platform, is Open\texttt{JNLP} at
\htmladdnormallink{http://openjnlp.nanode.org}{http://openjnlp.nanode.org}.
I haven't tried it. If you do, please update this document.)

\subsection{jar}

Jar is the standard java archive format. It basically uses the compression 
algorithm from zip with command line syntax similar to tar. If you need to 
uncompress a jar file on a machine with no jar, try your favorite unzip tool. 
Jar comes with the \texttt{JDK}.

\subsection{jarsigner}

Jars can be signed. Jars \textbf{must} be signed to work over Java Web Start.
But note that you can use ``test'' keys -- you don't actually have to
pay money to a certifying authority and get a ``real'' key that actually
``proves'' that you are who you say you are.

The jarsigner tool comes with the \texttt{JDK}. To use it, you need to create
a personal authentication key, and put it in a keystore file. The sign
targets in ``build.xml'' and ``Makefile'' assume that your key name and
keystore filename match your username. If you don't like this, you can
add new targets.

\begin{itemize}
\item[TODO:] Link to basic jarsigner docs.
\end{itemize}

\subsection{zip and unzip tools}

You need something that can work with the standard zip format. WinZip, 
Info-Zip, PKZip, etc. Info-Zip is free -- most of the others are
shareware. 

\subsection{Text editors}

Your choice. I recommend choosing something that autoconverts tabs to
the right number of spaces on the fly. (Tab characters in code are
\textbf{evil}) and can save files using Unix newline and \texttt{EOF}
conventions. If your editor can't do these, then you'll need to run
``ant fix'' on your code before checking it into \texttt{CVS}, to avoid
creating whole-file false diffs or files that look awful if someone has
different tab stop settings than yours. (Yes, cross-platform development
adds a few wrinkles.)

Among the nice editors for Unix, there's \hypercite{XEmacs}{XEmacs}{}{XEMACS}. XEmacs can use the
\hypercite{Java Development Environment for Emacs}{Java Development Environment for Emacs}{}{JDE}.

\section{Optional tools}

\subsection{Debuggers}

\texttt{JS}wat is free and pretty good.
\htmladdnormallink{http://www.bluemarsh.com/java/jswat/}{http://www.bluemarsh.com/java/jswat/}

\subsection{Profilers}

You can use \texttt{java -Xprof} or \texttt{-Xrunhprof}.
OptimizeIt! rocks, but it's \$500.
\begin{itemize}
\item[TODO:] Add ProfAnal info.
\end{itemize}

\subsection{Code reformatters}

If you just can't work with the format in
\htmladdnormallink{CodingStandards}{../CodingStandards/index.html},
then you need a tool to convert files to and from your personal pet
format to the one this project uses.

JIndent used to be free, but not that full-featured. 
\begin{itemize}
\item[TODO:] Find working link to last free version.
\end{itemize}
Now it's really full-featured, but no longer free.
\begin{itemize}
\item[TODO:] Find free Java code formatter that actually works well.
\end{itemize}

\subsection{Image editors}

Currently all images used in Colossus are \texttt{GIF}s or \texttt{PNG}s.
\texttt{JPEG}s are great for photos but not optimal for simple drawings.

Any image editor that can save to standard formats is fine. If you
don't have one, \hypercite{The Gimp}{The Gimp}{}{GIMP} is good and free,
but has a steep learning curve, much like Photoshop. 
\begin{itemize}
\item[TODO:] Find a lighter-weight, easier-to-learn free image editor that 
can save to our file formats.
\end{itemize}

\section{Build process}

Once you have the tools set up, Colossus basically builds itself.

First you need all the source files. Snag the latest zip file from 
\htmladdnormallink{http://colossus.sf.net/download/}{http://colossus.sf.net/download/},
or (to get the newest possible code) pull from \texttt{CVS}.

If you just type ``ant'' or ``make'' from the project base directory,
then you get the default target, which in both cases will compile
all the .jj files into .java files with JavaCC, then compile all
the .java files into .class files with javac, then make an unsigned
executable jar file.

Other interesting targets include ``clean'' (delete stuff, useful if
you want to clean up or make sure that you fully rebuild all your
class files), ``fix'' (ant only -- does a bit of text reformatting on
java source files), ``tools'' (builds standalone tools), ``sign'' (signs
a jarfile), and ``dist'' (makes a zip file). There are also some
targets that automate installing the package onto a web site --
these are designed for maintaining the \texttt{SF} site rather than general
use. Basically, skim through ``build.xml''.

\section{Gotchas}

\begin{itemize}
\item[TODO:] Find out what confuses people and fix or document it.
\end{itemize}

\addcontentsline{toc}{section}{References}
\bibliographystyle{plain}
\bibliography{build}

\end{document}
